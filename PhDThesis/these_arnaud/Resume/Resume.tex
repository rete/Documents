\chapter*{Résumé}

Le Collisionneur Linéaire International ILC est un projet de collisionneur électron-positon développé pour prendre le relais du Grand Collisionneur de Hadrons LHC. Ce projet permettra d'étudier précisément les caractéristiques du nouveau boson de 125 $GeV$, découvert en 2012 par les expérience CMS et ATLAS, compatible avec le boson de Higgs du modèle standard. Cette expérience pourrait aussi permettre aux physiciens de mettre à jour des phénomènes physiques inconnus. 

Pour exploiter au maximum ce nouvel accélérateur, deux collaborations travaillent sur le développement de deux détecteurs: le Grand Détecteur International ILD et le Détecteur au Silicium SiD. Ces détecteurs sont dits généralistes et sont optimisés pour la mise en œuvre de technique de suivi des particules. Ils sont constitués d'un trajectographe dans leur partie centrale et de systèmes de calorimétrie. L'ensemble est inséré dans un aimant supra-conducteur, lui même entouré d'une culasse instrumentée avec des chambres à muon.

Le groupe lyonnais dans lequel j'ai effectué mes travaux de recherche pendant mon doctorat, a grandement participé au développement du calorimètre hadronique à lecture semi-digitale. Ce calorimètre ultra-granulaire fait partie des options pour le calorimètre hadronique du Grand Détecteur International. Un prototype a été construit en 2011. D'environ 1 $m^3$, il est constitué de 48 chambres à plaque résistive de verre, comporte plus de 440000 canaux de lecture de 1 $cm^2$ et pèse environ 10 tonnes. Ce calorimètre répond aux contraintes imposées pour le Collisionneur Linaire International (une haute granularité, une consommation électrique faible, une alimentation pulsée etc) et est régulièrement testé sur des lignes de faisceau au CERN. 

Les données ainsi collectées m'ont permis d'étudier en détail le phénomène de gerbe hadronique. De nombreux efforts ont été réalisé pour développer des méthodes efficaces de reconstruction de l'énergie des gerbes hadroniques et pour améliorer la résolution en énergie du prototype SDHCAL. La simulation des gerbes hadroniques dans le SDHCAL constitue une part importante de mes travaux de recherche. Une simulation réaliste des chambres à plaque résistive de verre a été développée en étudiant la réponse du prototype au passage de muons et de gerbes électromagnétiques. J'ai alors confronté les modèles de simulation des gerbes hadroniques avec des données expérimentales. La granularité du SDHCAL rend aussi possible des études fines sur la topologie des gerbes hadroniques, notamment sur leur extension latérale et longitudinale.
 
Les prochaines étapes importantes du projet SDHCAL sont le développement de méthodes de suivi de particules dédiées à cette technologie et l'étude de ses performances avec des simulations de collisions électron-positon. 
