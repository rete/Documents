\chapter*{Conclusions et perspectives}

Le prototype de calorimètre hadronique semi-digital construit en 2011, en grande partie au sein de l'Institut de Physique Nucléaire de Lyon, est depuis régulièrement testé sur les lignes de faisceau du CERN. Les résultats obtenus avec prototype de calorimètre ultra-granulaire sont très encourageants. L'énergie des gerbes hadroniques y est mesurée précisément. Des nouvelles techniques de calibration, afin d'augmenter homogénéïté de la réponse du détecteur, devraient permettre des améliorations de ces mesures. 

Une simulation détaillée du prototype a été développée. Son paramétrage s'est appuyé sur la réponse du détecteur aux passages de muons et de gerbes électromagnétiques. Cette simulation a ensuite permis de tester différents modèles de simulation, développés par la collaboration GEANT4. Des différences significatives sur la réponse du détecteur entre les données expérimentales et la plupart des modèles de simulation ont été observées. Dans le but de comprendre ces différences et de discriminer les modèles, les profils longitudinal et latéral des gerbes hadroniques ont été comparés entre les données et les simulations. Enfin, une technique robuste de reconstruction des traces a été mise en oeuvre. Cette méthode a été, dans un premier temps, testée avec les événements muons, puis utilisée pour comparer les simulations de gerbes hadroniques.

Enfin, les performances du calorimètre hadronique semi-digital ont été étudiées dans une simulation complète du Grand Détecteur International. Les études de reconstruction de l'énergie des jets ont notamment permis de valider le concept de suivi de particules. L'études des réactions, produisant des paires de boson W et Z, se désintégrant hadroniquement, montre une bonne séparation de la masse de ces bosons.

Prochainement, des nouvelles chambres à plaque résistive de verre, de 2 $m^2$, seront construites afin de valider la faisabilité de telles chambres pour le détecteur final. En parallèle, des techniques d'analyses multivariés serons développées pour essayer de reconstruire plus précisément l'énergie des cascades hadroniques dans le SDHCAL. Un algorithme de suivi des particules, en cours de développement, sera testé avec une simulation complète du Grand Détecteur International. Enfin, les performances de ce détecteur avec le calorimètre semi-digital pour l'étude du boson de Higgs, particulièrement à travers la réaction $e^+e^-\rightarrow ZH$, seront prochainement étudiées.

