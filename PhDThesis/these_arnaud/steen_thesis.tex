\documentclass[12pt]{book}

\setlength{\hoffset}{-1in}
\setlength{\voffset}{-1in}
\setlength{\topmargin}{1.5cm}
\setlength{\headheight}{1.cm}
\setlength{\headsep}{.5cm}%zone de texte header)
\setlength{\topskip}{0cm}
\setlength{\textheight}{23cm}
\setlength{\footskip}{3cm}
\setlength{\oddsidemargin}{3.25cm}
\setlength{\evensidemargin}{3.25cm}
\setlength{\textwidth}{15cm}

%%%%%%%%%%%%%%%%%%%%%%%%%%%%%%%%%%%%

\usepackage[frenchb]{babel}
\usepackage[utf8]{inputenc}
\usepackage{pifont}
\usepackage[T1]{fontenc}
\usepackage{verbatim}
\usepackage[nooneline]{subfigure}
\usepackage{graphicx}
\usepackage{dcolumn}
\usepackage{bm}
\usepackage{mathrsfs}
\usepackage{amsmath}
\usepackage{amssymb}
\usepackage{gensymb}
\usepackage{textcomp}
\usepackage{wasysym}
\usepackage{enumerate}
\usepackage{appendix}
\usepackage{lscape}

%Pour les petit table of content en début de chapitre
\usepackage{minitoc}
\mtcsettitle{minitoc}{Contenu}
%Profondeur d'indexage
%\setcounter{secnumdepth}{4}
%\setcounter{minitocdepth}{2}

%\usepackage{multicol,caption}
\usepackage{caption}
%\usepackage{subcaption}
\usepackage{amsmath}
\usepackage{lineno} % Numéro de lignes
\usepackage{ upgreek } %Lettres grec grandes
\usepackage{textcomp}
\usepackage[colorlinks= false, urlcolor=blue, citebordercolor={0 1 0}, linkbordercolor ={1 0 0},pdfborder={0 0 0}]{hyperref} %Gestion des affichages (Citations, Acronymes)
\usepackage[printonlyused,withpage]{acronym}  %Utilisation des acronymes

\usepackage[version=3]{mhchem} %Pour les élements chimiques

\usepackage{multirow}
\usepackage{multicol}
\usepackage{colortbl}
\usepackage[table]{xcolor}
\usepackage{array}
\usepackage{fancyhdr}
\usepackage{perpage}
\MakePerPage{footnote}
\captionsetup{font=sl}

\renewcommand{\contentsname}{Table des matières}

%%%%%%%%%%%%%%%%%%%%%%%%%%%%%%%%%%
%%Déclaration de toutes mes commandes   %%%%%%%%%%%%
%%%%%%%%%%%%%%%%%%%%%%%%%%%%%%%%%%
\input Fonctions/myCommands.tex
\input Fonctions/ltxdefs.tex



%%%%%%%%%%%%%%%%%%%%%%%%%%%%%%%%%
% Page de couverture %%%%%%%%%%%%%%%%%%%%%%
%%%%%%%%%%%%%%%%%%%%%%%%%%%%%%%%%
\makeatletter

\def\clap#1{\hbox to 0pt{\hss #1\hss}}%

\def\ligne#1{
  \hbox to \hsize{
    \vbox{\centering #1}}}

\def\haut#1#2#3{
  \hbox to \hsize{
    \rlap{\vtop{\raggedright #1}}
    \hss
    \clap{\vtop{\centering #2}}
    \hss
    \llap{\vtop{\raggedleft #3}}}}

\def\bas#1#2#3{
  \hbox to \hsize{
    \rlap{\vbox{\raggedright #1}}
    \hss
    \clap{\vbox{\centering #2}}
    \hss
    \llap{\vbox{\raggedleft #3}}}}

% Titre
\def\maketitle{
  \thispagestyle{empty}\vbox to \vsize{
    \vfill
    \haut{}{\large \@blurb}{}
    \vfill
    \ligne{\LARGE \@title}
    \vspace{5mm}
    \ligne{\LARGE \@author}
    \vfill
    \ligne{\Large \@jury}
    \vfill
  }
}

% Date, lieu,...
\def\date#1{\def\@date{#1}}
\def\author#1{\def\@author{#1}}
\def\title#1{\def\@title{#1}}
\def\location#1{\def\@location{#1}}
\def\blurb#1{\def\@blurb{#1}}
\def\jury#1{\def\@jury{#1}}
\def\email#1{\def\@email{#1}}
\def\administratif#1{\def\@administratif{#1}}
%\def\includegraphics#1{\def\@includegraphics{#1}}

% Valeurs par defaut
\date{\today}
\author{}
\title{}
\location{Villeurbanne}
\blurb{}

\email{steen@ipnl.in2p3.fr}

\makeatother
\title{\Huge Étude des gerbes hadroniques à l’aide du prototype du calorimètre hadronique semi-digitial et comparaison avec les modèles théoriques utilisés dans le logiciel GEANT4}
\author{Arnaud \textsc{Steen}}
\location{Villeurbanne}
\administratif{
  %Sous la direction de :\\
  %Imad  \textsc{Laktineh} et 
}
\blurb{
  Th\`ese de l'Universit\'e de Lyon \\
  \bigskip
  \'Ecole Doctorale de Physique et Astrophysique de Lyon (PHAST)\\[1em]
  \bigskip
  Th\`ese de Doctorat en Physique des Particules \\
  \bigskip
  Institut de Physique Nucl\'eaire de Lyon (IPNL)\\
}
\jury{
  \begin{center}
    \begin{tabular}{lll}
      JURY~: & M.  Didier \textsc{Contardo} & (examinateur)\\
      & M.  Gérald \textsc{Grenier} & (co-directeur de th\`ese)\\
      & M.  Imad  \textsc{Laktineh} & (directeur de th\`ese)\\
      & M.  Dominique \textsc{Pallin}       & (rapporteur)\\
      & M.  Alberto \textsc{Ribon}       & (examinateur)\\
      & M.  Laurent \textsc{Serin}      & (rapporteur)\\
    \end{tabular}
  \end{center}
}
%%%%%%%%%%%%%%%%%%%%%%%%%%%%%%%

\interfootnotelinepenalty=10000


\begin{document}
\begin{figure}
  \begin{center}
    \includegraphics[width=0.18\textwidth]{Logo/logo_white_ILC.jpg}
    \hspace{1cm}
    \includegraphics[width=0.18\textwidth]{Logo/calice_logo.jpg}
    \hspace{1cm}
    \includegraphics[width=0.18\textwidth]{Logo/Logo_IPNL.jpg}
    \hspace{1cm}
    \includegraphics[width=0.18\textwidth]{Logo/logo_ucbl_lyon1.jpg}
  \end{center}
\end{figure}

\maketitle{\thispagestyle{empty}}

\clearpage
%\pageblanche

\pagestyle{fancy}
\fancyhf{}
\renewcommand{\sectionmark}[1]{ \markright{\thesection\, #1}}
\renewcommand{\chaptermark}[1]{%
\markboth{\sc\thechapter.\ #1}{}}

\lhead[\thepage]{\rightmark}
\rhead[\leftmark]{\thepage}

\dominitoc
\modulolinenumbers[5]
\linenumbers

\pagestyle{empty}
\input Resume/Resume.tex
\pageblanche

\frontmatter %active les numérot de pages en i
\tableofcontents{\thispagestyle{empty}}
\pageblanche
%\pageblanche
\cleardoublepage


%%%%%%%%%%%%%%%%%%%%%%%%%%%%%%%%%
%Activation numéro de lignes            %%%%%%%%%%%%%%%
%%%%%%%%%%%%%%%%%%%%%%%%%%%%%%%%%
%\setpagewiselinenumbers


\renewcommand\listfigurename{Liste des figures}
\listoffigures{\thispagestyle{empty}}

%%###############################################
%%#######Liste des Tables  ########################
%%###############################################
\listoftables{\thispagestyle{empty}}

%On désactive ici les haut de pages
\pagestyle{empty}

%###############################################
%%#######MicroIntro###############
%%###############################################
\renewcommand*\thesection{\arabic{section}}
\input Introduction/Introduction.tex

\pagestyle{fancy}
\fancyhf{}
\lhead[\thepage]{\rightmark}
 \rhead[\leftmark]{\thepage}

\renewcommand*\thesection{\arabic{section}}
\mainmatter
%\def\etc{\latinabbrev{etc}}
\input ModelStandard/ModelStandard.tex
\input ILC/ILC.tex
\input ShowerTh/ShowerTh.tex
\input SDHCAL/SDHCAL.tex
\input Digitizer/Digitizer.tex
\input Shower/Shower.tex
\input ILD/ILD.tex
%\backmatter
%###############################################
%#######Bibliographie #############################
%###############################################
\bibliography{Bibliography/Bibliography}
\bibliographystyle{Bibliography/MyRefStyle}
\addcontentsline{toc}{chapter}{\bibname} 

\end{document}
