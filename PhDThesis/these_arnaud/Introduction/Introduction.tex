\chapter*{Introduction}

Le calorimètre à lecture semi-digitale SDHCAL est un détecteur ultra-granulaire, proposé pour équiper le calorimètre hadronique du Grand Détecteur International ILD du futur Collisionneur Linéaire International ILC. Ce projet de collisionneur électron-positon a été imaginé par la communauté scientifique afin de poursuivre le programme de recherche du Grand Collisionneur de Hadron LHC. Un prototype de calorimètre à lecture semi-digitale a été construit en grande partie au sein de l'Institut de Physique Nucléaire de Lyon en 2011. Il a depuis été testé lors de plusieurs campagnes de tests sur faisceau au CERN. Ce manuscrit synthétise les travaux de recherche que j'ai réalisé sur les gerbes hadroniques à l'aide du prototype SDHCAL. 
\\ \\
Le chapitre~\ref{chap.ms} introduit le contexte théorique de la physique des particules: le modèle standard. Nous introduirons les différents mécanismes d'interaction entre les constituants de la matière dans le cadre de ce modèle. Puis nous discuterons des limites du modèle standard, qui sont des motivations pour le développement de nouveaux accélérateurs.
\\ \\
Dans le chapitre~\ref{chap.ilc} nous présenterons le Collisionneur Linéaire International, candidat au statut d'expérience majeure de physique des particules. Nous nous intéresserons aux détecteurs imaginés pour enregistrer les événements de collision auprès de l'ILC et plus particulièrement ceux du Grand Détecteur International pour lequel le SDHCAL est proposé. Nous discuterons aussi briévement des autres projets de collisionneur leptonique.
\\ \\
Le chapitre~\ref{chap.shower} sera dédié à la description du phénomène de gerbe électromagnétique ou hadronique, principal objet d'étude de ce manuscrit. Nous présenterons les nombreuses formes d'interaction des particules dans la matière, conduisant au phénomène de cascade électromagnétique ou hadronique.
\\ \\
Nous décrirons en détail le prototype de calorimètre hadronique à lecture semi-digitale SDHCAL dans le chapitre~\ref{chap.sdhcal}. Nous insisterons particulièrement sur la description des chambres à plaque résistive de verre qui composent la partie active de ce détecteur. Enfin, nous détaillerons les méthodes utilisées pour reconstruire l'énergie des hadrons dans ce calorimètre et nous présenterons les résultats ainsi obtenus.
\\ \\
Le chapitre~\ref{chap.simulation} sera consacré au développement de la simulation du prototype SDHCAL. Nous présenterons rapidement les modèles théoriques utilisés par la simulation, puis nous décrirons les différentes méthodes développées pour obtenir une simulation la plus réaliste possible. Ces méthodes se sont principalement appuyées sur l'étude de la réponse des chambres à plaque résistive de verre, lors du passage de muons ou de gerbes électromagnétiques dans le SDHCAL. Enfin, nous exposerons les comparaisons entre les données obtenues lors des tests en faisceau et les modèles de simulations sur la réponse du prototype SDHCAL.
\\ \\
Ensuite, nous étudierons la topologie des gerbes hadroniques dans le chapitre~\ref{chap.topo}. Les profils longitudinal et transversal des gerbes hadroniques, les traces reconstruites dans ces cascades, seront examinés grâce à la granularité très fine du SDHCAL. Ces variables nous permettront de discriminer les différents modèles de simulations.
\\ \\
Enfin, dans le chapitre~\ref{chap.ild}, nous étudierons les performances du Grand Détecteur International (ILD) avec le calorimètre hadronique à lecture semi-digitale. La résolution en énergie des jets et la reconstruction de la masse des bosons $W$ et $Z$ seront développées.
